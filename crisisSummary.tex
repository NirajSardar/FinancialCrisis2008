% Options for packages loaded elsewhere
\PassOptionsToPackage{unicode}{hyperref}
\PassOptionsToPackage{hyphens}{url}
%
\documentclass[
]{article}
\usepackage{amsmath,amssymb}
\usepackage{lmodern}
\usepackage{iftex}
\ifPDFTeX
  \usepackage[T1]{fontenc}
  \usepackage[utf8]{inputenc}
  \usepackage{textcomp} % provide euro and other symbols
\else % if luatex or xetex
  \usepackage{unicode-math}
  \defaultfontfeatures{Scale=MatchLowercase}
  \defaultfontfeatures[\rmfamily]{Ligatures=TeX,Scale=1}
\fi
% Use upquote if available, for straight quotes in verbatim environments
\IfFileExists{upquote.sty}{\usepackage{upquote}}{}
\IfFileExists{microtype.sty}{% use microtype if available
  \usepackage[]{microtype}
  \UseMicrotypeSet[protrusion]{basicmath} % disable protrusion for tt fonts
}{}
\makeatletter
\@ifundefined{KOMAClassName}{% if non-KOMA class
  \IfFileExists{parskip.sty}{%
    \usepackage{parskip}
  }{% else
    \setlength{\parindent}{0pt}
    \setlength{\parskip}{6pt plus 2pt minus 1pt}}
}{% if KOMA class
  \KOMAoptions{parskip=half}}
\makeatother
\usepackage{xcolor}
\usepackage[margin=1in]{geometry}
\usepackage{color}
\usepackage{fancyvrb}
\newcommand{\VerbBar}{|}
\newcommand{\VERB}{\Verb[commandchars=\\\{\}]}
\DefineVerbatimEnvironment{Highlighting}{Verbatim}{commandchars=\\\{\}}
% Add ',fontsize=\small' for more characters per line
\usepackage{framed}
\definecolor{shadecolor}{RGB}{248,248,248}
\newenvironment{Shaded}{\begin{snugshade}}{\end{snugshade}}
\newcommand{\AlertTok}[1]{\textcolor[rgb]{0.94,0.16,0.16}{#1}}
\newcommand{\AnnotationTok}[1]{\textcolor[rgb]{0.56,0.35,0.01}{\textbf{\textit{#1}}}}
\newcommand{\AttributeTok}[1]{\textcolor[rgb]{0.77,0.63,0.00}{#1}}
\newcommand{\BaseNTok}[1]{\textcolor[rgb]{0.00,0.00,0.81}{#1}}
\newcommand{\BuiltInTok}[1]{#1}
\newcommand{\CharTok}[1]{\textcolor[rgb]{0.31,0.60,0.02}{#1}}
\newcommand{\CommentTok}[1]{\textcolor[rgb]{0.56,0.35,0.01}{\textit{#1}}}
\newcommand{\CommentVarTok}[1]{\textcolor[rgb]{0.56,0.35,0.01}{\textbf{\textit{#1}}}}
\newcommand{\ConstantTok}[1]{\textcolor[rgb]{0.00,0.00,0.00}{#1}}
\newcommand{\ControlFlowTok}[1]{\textcolor[rgb]{0.13,0.29,0.53}{\textbf{#1}}}
\newcommand{\DataTypeTok}[1]{\textcolor[rgb]{0.13,0.29,0.53}{#1}}
\newcommand{\DecValTok}[1]{\textcolor[rgb]{0.00,0.00,0.81}{#1}}
\newcommand{\DocumentationTok}[1]{\textcolor[rgb]{0.56,0.35,0.01}{\textbf{\textit{#1}}}}
\newcommand{\ErrorTok}[1]{\textcolor[rgb]{0.64,0.00,0.00}{\textbf{#1}}}
\newcommand{\ExtensionTok}[1]{#1}
\newcommand{\FloatTok}[1]{\textcolor[rgb]{0.00,0.00,0.81}{#1}}
\newcommand{\FunctionTok}[1]{\textcolor[rgb]{0.00,0.00,0.00}{#1}}
\newcommand{\ImportTok}[1]{#1}
\newcommand{\InformationTok}[1]{\textcolor[rgb]{0.56,0.35,0.01}{\textbf{\textit{#1}}}}
\newcommand{\KeywordTok}[1]{\textcolor[rgb]{0.13,0.29,0.53}{\textbf{#1}}}
\newcommand{\NormalTok}[1]{#1}
\newcommand{\OperatorTok}[1]{\textcolor[rgb]{0.81,0.36,0.00}{\textbf{#1}}}
\newcommand{\OtherTok}[1]{\textcolor[rgb]{0.56,0.35,0.01}{#1}}
\newcommand{\PreprocessorTok}[1]{\textcolor[rgb]{0.56,0.35,0.01}{\textit{#1}}}
\newcommand{\RegionMarkerTok}[1]{#1}
\newcommand{\SpecialCharTok}[1]{\textcolor[rgb]{0.00,0.00,0.00}{#1}}
\newcommand{\SpecialStringTok}[1]{\textcolor[rgb]{0.31,0.60,0.02}{#1}}
\newcommand{\StringTok}[1]{\textcolor[rgb]{0.31,0.60,0.02}{#1}}
\newcommand{\VariableTok}[1]{\textcolor[rgb]{0.00,0.00,0.00}{#1}}
\newcommand{\VerbatimStringTok}[1]{\textcolor[rgb]{0.31,0.60,0.02}{#1}}
\newcommand{\WarningTok}[1]{\textcolor[rgb]{0.56,0.35,0.01}{\textbf{\textit{#1}}}}
\usepackage{graphicx}
\makeatletter
\def\maxwidth{\ifdim\Gin@nat@width>\linewidth\linewidth\else\Gin@nat@width\fi}
\def\maxheight{\ifdim\Gin@nat@height>\textheight\textheight\else\Gin@nat@height\fi}
\makeatother
% Scale images if necessary, so that they will not overflow the page
% margins by default, and it is still possible to overwrite the defaults
% using explicit options in \includegraphics[width, height, ...]{}
\setkeys{Gin}{width=\maxwidth,height=\maxheight,keepaspectratio}
% Set default figure placement to htbp
\makeatletter
\def\fps@figure{htbp}
\makeatother
\setlength{\emergencystretch}{3em} % prevent overfull lines
\providecommand{\tightlist}{%
  \setlength{\itemsep}{0pt}\setlength{\parskip}{0pt}}
\setcounter{secnumdepth}{-\maxdimen} % remove section numbering
\ifLuaTeX
  \usepackage{selnolig}  % disable illegal ligatures
\fi
\IfFileExists{bookmark.sty}{\usepackage{bookmark}}{\usepackage{hyperref}}
\IfFileExists{xurl.sty}{\usepackage{xurl}}{} % add URL line breaks if available
\urlstyle{same} % disable monospaced font for URLs
\hypersetup{
  pdftitle={Financial Crisis},
  pdfauthor={Niraj Sardar},
  hidelinks,
  pdfcreator={LaTeX via pandoc}}

\title{Financial Crisis}
\author{Niraj Sardar}
\date{}

\begin{document}
\maketitle

We are provided with a csv file containing annual financial data for
firms. We are going to utilize 3 columns name fyear = fiscal year; tic =
firm ID, ni = net income.

\hypertarget{the-effect-of-the-crisis-in-financial-terms-.}{%
\subsection{The effect of the crisis in financial terms
.}\label{the-effect-of-the-crisis-in-financial-terms-.}}

For this we will be calculating the percent by which average annual net
income decreases for firms during crisis years (2007-8) compared to
pre-crisis years (2004-6)

Step 1 : Load the data and filter it .

\begin{Shaded}
\begin{Highlighting}[]
\FunctionTok{library}\NormalTok{(dplyr)}
\end{Highlighting}
\end{Shaded}

\begin{verbatim}
## 
## Attaching package: 'dplyr'
\end{verbatim}

\begin{verbatim}
## The following objects are masked from 'package:stats':
## 
##     filter, lag
\end{verbatim}

\begin{verbatim}
## The following objects are masked from 'package:base':
## 
##     intersect, setdiff, setequal, union
\end{verbatim}

\begin{Shaded}
\begin{Highlighting}[]
\NormalTok{data }\OtherTok{=} \FunctionTok{read.csv}\NormalTok{(}\StringTok{"V:/bkp N/FINC/Week4/Compustat 1990{-}2015 Lots.csv"}\NormalTok{)}
\NormalTok{data }\OtherTok{\textless{}{-}}\NormalTok{ data }\SpecialCharTok{\%\textgreater{}\%} \FunctionTok{select}\NormalTok{(}\StringTok{\textquotesingle{}fyear\textquotesingle{}}\NormalTok{,}\StringTok{\textquotesingle{}tic\textquotesingle{}}\NormalTok{,}\StringTok{\textquotesingle{}ni\textquotesingle{}}\NormalTok{)}
\NormalTok{dataForCrisis }\OtherTok{\textless{}{-}}\NormalTok{ data }\SpecialCharTok{\%\textgreater{}\%} \FunctionTok{filter}\NormalTok{(ni }\SpecialCharTok{!=} \StringTok{\textquotesingle{}NA\textquotesingle{}}\NormalTok{, ni }\SpecialCharTok{!=} \DecValTok{0}\NormalTok{ ,fyear }\SpecialCharTok{\textgreater{}=} \DecValTok{2004} \SpecialCharTok{\&}\NormalTok{ fyear }\SpecialCharTok{\textless{}=} \DecValTok{2008}\NormalTok{) }

\NormalTok{dataForCrisis }\OtherTok{\textless{}{-}}\NormalTok{ dataForCrisis }\SpecialCharTok{\%\textgreater{}\%} \FunctionTok{group\_by}\NormalTok{(dataForCrisis}\SpecialCharTok{$}\NormalTok{tic) }\SpecialCharTok{\%\textgreater{}\%} \FunctionTok{filter}\NormalTok{(}\FunctionTok{all}\NormalTok{(}\FunctionTok{c}\NormalTok{(}\DecValTok{2004}\SpecialCharTok{:}\DecValTok{2008} \SpecialCharTok{\%in\%}\NormalTok{ dataForCrisis}\SpecialCharTok{$}\NormalTok{fyear))) }\SpecialCharTok{\%\textgreater{}\%}\NormalTok{ ungroup}
\end{Highlighting}
\end{Shaded}

Step 2 :I have seperated the data in two data frames for pre-crisis and
post crisis . After summerising the data I will merge both of them to
calculate percentage change. For summarising the data we are calculating
the annual net average income for the pre crisis and post crisis which
are denoted by `ANIPRE' and `ANIPOST' respectively.

\begin{Shaded}
\begin{Highlighting}[]
\NormalTok{dfPre }\OtherTok{\textless{}{-}}\NormalTok{ dataForCrisis }\SpecialCharTok{\%\textgreater{}\%} \FunctionTok{filter}\NormalTok{(fyear }\SpecialCharTok{\textgreater{}=} \DecValTok{2004} \SpecialCharTok{\&}\NormalTok{ fyear }\SpecialCharTok{\textless{}=} \DecValTok{2006}\NormalTok{)}
\NormalTok{PreCrisisData  }\OtherTok{\textless{}{-}}\NormalTok{ dfPre }\SpecialCharTok{\%\textgreater{}\%} \FunctionTok{group\_by}\NormalTok{(tic) }\SpecialCharTok{\%\textgreater{}\%} \FunctionTok{summarise}\NormalTok{(}\AttributeTok{ANIPRE =}\FunctionTok{mean}\NormalTok{(ni))}
\FunctionTok{head}\NormalTok{(PreCrisisData,}\DecValTok{5}\NormalTok{)}
\end{Highlighting}
\end{Shaded}

\begin{verbatim}
## # A tibble: 5 x 2
##   tic      ANIPRE
##   <chr>     <dbl>
## 1 ""        -2.03
## 2 "0015B"   41.7 
## 3 "0030B"   19.4 
## 4 "0032A"  660   
## 5 "0033A" 1305
\end{verbatim}

\begin{Shaded}
\begin{Highlighting}[]
\NormalTok{dfPost  }\OtherTok{\textless{}{-}}\NormalTok{ dataForCrisis }\SpecialCharTok{\%\textgreater{}\%} \FunctionTok{filter}\NormalTok{(fyear }\SpecialCharTok{\textgreater{}=} \DecValTok{2007} \SpecialCharTok{\&}\NormalTok{ fyear }\SpecialCharTok{\textless{}=} \DecValTok{2008}\NormalTok{)}
\NormalTok{PostCrisisData  }\OtherTok{\textless{}{-}}\NormalTok{ dfPost }\SpecialCharTok{\%\textgreater{}\%} \FunctionTok{group\_by}\NormalTok{(tic) }\SpecialCharTok{\%\textgreater{}\%} \FunctionTok{summarise}\NormalTok{(}\AttributeTok{ANIPOST =}\FunctionTok{mean}\NormalTok{(ni) )}
\FunctionTok{head}\NormalTok{(PostCrisisData,}\DecValTok{5}\NormalTok{)}
\end{Highlighting}
\end{Shaded}

\begin{verbatim}
## # A tibble: 5 x 2
##   tic      ANIPOST
##   <chr>      <dbl>
## 1 ""         -3.14
## 2 "0015B"  -120.  
## 3 "0030B"     9.87
## 4 "0032A" -4500.  
## 5 "0033A" -5238.
\end{verbatim}

Step 3 :Merging the two datframes of pre and post crisis and calculating
the percentage change

\begin{Shaded}
\begin{Highlighting}[]
\NormalTok{finalCrisDf }\OtherTok{\textless{}{-}} \FunctionTok{inner\_join}\NormalTok{(PreCrisisData , PostCrisisData , }\AttributeTok{by =} \FunctionTok{c}\NormalTok{(}\StringTok{"tic"}\NormalTok{))}
\NormalTok{finalCrisDf}\SpecialCharTok{$}\NormalTok{pchng }\OtherTok{=}\NormalTok{ ((finalCrisDf}\SpecialCharTok{$}\NormalTok{ANIPOST }\SpecialCharTok{{-}}\NormalTok{ finalCrisDf}\SpecialCharTok{$}\NormalTok{ANIPRE)}\SpecialCharTok{/}\FunctionTok{abs}\NormalTok{(finalCrisDf}\SpecialCharTok{$}\NormalTok{ANIPRE))}\SpecialCharTok{*}\DecValTok{100}
\FunctionTok{head}\NormalTok{(finalCrisDf,}\DecValTok{5}\NormalTok{) }
\end{Highlighting}
\end{Shaded}

\begin{verbatim}
## # A tibble: 5 x 4
##   tic      ANIPRE  ANIPOST  pchng
##   <chr>     <dbl>    <dbl>  <dbl>
## 1 ""        -2.03    -3.14  -54.6
## 2 "0015B"   41.7   -120.   -389. 
## 3 "0030B"   19.4      9.87  -49.0
## 4 "0032A"  660    -4500.   -782. 
## 5 "0033A" 1305    -5238.   -501.
\end{verbatim}

Analysis : Based on the values it can be concluded that the net income
has a very steep decline in its value . This crisis was result of crash
of stock market which expanded globally . Net income contributes to a
company's assets and can therefore affect the book value . Thus it can
be concluded that due to crisis there must be certain cost cutting which
multiple firms might have considered like firing employees, shutting
down offices , changing the value of utilities , changing cost price and
affecting the real estates business .

\hypertarget{calculate-the-absolute-difference-absni-of-average-annual-net-income-between-pre-crisis-years-and-crisis-years.}{%
\subsection{Calculate the absolute difference (AbsNi) of average annual
net income between pre-crisis years and crisis
years.}\label{calculate-the-absolute-difference-absni-of-average-annual-net-income-between-pre-crisis-years-and-crisis-years.}}

\begin{Shaded}
\begin{Highlighting}[]
\NormalTok{dataMergeAbs }\OtherTok{\textless{}{-}} \FunctionTok{inner\_join}\NormalTok{(PreCrisisData , PostCrisisData , }\AttributeTok{by =} \FunctionTok{c}\NormalTok{(}\StringTok{"tic"}\NormalTok{)) }
\NormalTok{dataMergeAbs}\SpecialCharTok{$}\NormalTok{AbsNi }\OtherTok{=}  \FunctionTok{abs}\NormalTok{(dataMergeAbs}\SpecialCharTok{$}\NormalTok{ANIPOST }\SpecialCharTok{{-}}\NormalTok{ dataMergeAbs}\SpecialCharTok{$}\NormalTok{ANIPRE)}
\FunctionTok{head}\NormalTok{(dataMergeAbs,}\DecValTok{5}\NormalTok{)}
\end{Highlighting}
\end{Shaded}

\begin{verbatim}
## # A tibble: 5 x 4
##   tic      ANIPRE  ANIPOST   AbsNi
##   <chr>     <dbl>    <dbl>   <dbl>
## 1 ""        -2.03    -3.14    1.11
## 2 "0015B"   41.7   -120.    162.  
## 3 "0030B"   19.4      9.87    9.49
## 4 "0032A"  660    -4500.   5160.  
## 5 "0033A" 1305    -5238.   6542.
\end{verbatim}

\hypertarget{calculate-the-absolute-value-of-largest-and-smallest-changes-in-net-income-during-pre--crisis-and-crisis-period-below}{%
\subsection{Calculate the absolute value of largest and smallest changes
in Net Income during pre- crisis and crisis period below
:}\label{calculate-the-absolute-value-of-largest-and-smallest-changes-in-net-income-during-pre--crisis-and-crisis-period-below}}

\begin{Shaded}
\begin{Highlighting}[]
\NormalTok{dataAbsLargestChanges }\OtherTok{\textless{}{-}}\NormalTok{ dataMergeAbs }\SpecialCharTok{\%\textgreater{}\%} \FunctionTok{arrange}\NormalTok{(}\FunctionTok{desc}\NormalTok{(AbsNi)) }\SpecialCharTok{\%\textgreater{}\%} \FunctionTok{select}\NormalTok{(tic,AbsNi) }\SpecialCharTok{\%\textgreater{}\%} \FunctionTok{top\_n}\NormalTok{(}\DecValTok{10}\NormalTok{)}
\end{Highlighting}
\end{Shaded}

\begin{verbatim}
## Selecting by AbsNi
\end{verbatim}

\begin{Shaded}
\begin{Highlighting}[]
\FunctionTok{print}\NormalTok{(dataAbsLargestChanges)}
\end{Highlighting}
\end{Shaded}

\begin{verbatim}
## # A tibble: 10 x 2
##    tic    AbsNi
##    <chr>  <dbl>
##  1 AIG   57999.
##  2 FNMA  35503.
##  3 C     33091.
##  4 GM    31549 
##  5 VOD   29826.
##  6 FMCC  29032.
##  7 BAC2  23378.
##  8 UBS   21383.
##  9 RBS   19925.
## 10 S     16889.
\end{verbatim}

\begin{Shaded}
\begin{Highlighting}[]
\NormalTok{dataAbsSmallestChanges }\OtherTok{\textless{}{-}}\NormalTok{ dataMergeAbs }\SpecialCharTok{\%\textgreater{}\%} \FunctionTok{arrange}\NormalTok{(AbsNi) }\SpecialCharTok{\%\textgreater{}\%} \FunctionTok{select}\NormalTok{(tic,AbsNi) }\SpecialCharTok{\%\textgreater{}\%} \FunctionTok{top\_n}\NormalTok{(}\SpecialCharTok{{-}}\DecValTok{10}\NormalTok{)}
\end{Highlighting}
\end{Shaded}

\begin{verbatim}
## Selecting by AbsNi
\end{verbatim}

\begin{Shaded}
\begin{Highlighting}[]
\FunctionTok{print}\NormalTok{(dataAbsSmallestChanges)}
\end{Highlighting}
\end{Shaded}

\begin{verbatim}
## # A tibble: 10 x 2
##    tic       AbsNi
##    <chr>     <dbl>
##  1 MDJT.1 0       
##  2 AMRB   0.000167
##  3 ASOE   0.000333
##  4 CEHC   0.000833
##  5 VSYS   0.00100 
##  6 ANML.1 0.00117 
##  7 VODG   0.00117 
##  8 IHT    0.00133 
##  9 ORRMF  0.00233 
## 10 AXRX   0.0035
\end{verbatim}

\hypertarget{calculate-the-largest-and-smallest-percentage-changes-in-net-income-during-pre--crisis-and-crisis-period.}{%
\subsection{Calculate the largest and smallest percentage changes in Net
Income during pre- crisis and crisis
period.}\label{calculate-the-largest-and-smallest-percentage-changes-in-net-income-during-pre--crisis-and-crisis-period.}}

\begin{enumerate}
\def\labelenumi{\alph{enumi})}
\tightlist
\item
  Considering absolute value of percentage change(pchng) considering
  both increase and decrease in net income the top 10 largest and
  smallest Percentage values of firms are calculated respectively below
  .
\end{enumerate}

\begin{Shaded}
\begin{Highlighting}[]
\NormalTok{finalCrisDf2 }\OtherTok{\textless{}{-}} \FunctionTok{inner\_join}\NormalTok{(PreCrisisData , PostCrisisData , }\AttributeTok{by =} \FunctionTok{c}\NormalTok{(}\StringTok{"tic"}\NormalTok{))}

\NormalTok{finalCrisDf2}\SpecialCharTok{$}\NormalTok{pchng }\OtherTok{=} \FunctionTok{abs}\NormalTok{(((finalCrisDf2}\SpecialCharTok{$}\NormalTok{ANIPOST }\SpecialCharTok{{-}}\NormalTok{ finalCrisDf2}\SpecialCharTok{$}\NormalTok{ANIPRE)}\SpecialCharTok{/} \FunctionTok{abs}\NormalTok{(finalCrisDf2}\SpecialCharTok{$}\NormalTok{ANIPRE))}\SpecialCharTok{*}\DecValTok{100}\NormalTok{)}

\FunctionTok{head}\NormalTok{(finalCrisDf2,}\DecValTok{3}\NormalTok{)}
\end{Highlighting}
\end{Shaded}

\begin{verbatim}
## # A tibble: 3 x 4
##   tic     ANIPRE ANIPOST pchng
##   <chr>    <dbl>   <dbl> <dbl>
## 1 ""       -2.03   -3.14  54.6
## 2 "0015B"  41.7  -120.   389. 
## 3 "0030B"  19.4     9.87  49.0
\end{verbatim}

\begin{Shaded}
\begin{Highlighting}[]
\NormalTok{LargestAbsPerChng }\OtherTok{\textless{}{-}}\NormalTok{ finalCrisDf2 }\SpecialCharTok{\%\textgreater{}\%} \FunctionTok{arrange}\NormalTok{(pchng) }\SpecialCharTok{\%\textgreater{}\%} \FunctionTok{select}\NormalTok{(tic,pchng) }\SpecialCharTok{\%\textgreater{}\%} \FunctionTok{top\_n}\NormalTok{(}\DecValTok{10}\NormalTok{)}
\end{Highlighting}
\end{Shaded}

\begin{verbatim}
## Selecting by pchng
\end{verbatim}

\begin{Shaded}
\begin{Highlighting}[]
\FunctionTok{head}\NormalTok{(LargestAbsPerChng,}\DecValTok{10}\NormalTok{)}
\end{Highlighting}
\end{Shaded}

\begin{verbatim}
## # A tibble: 10 x 2
##    tic      pchng
##    <chr>    <dbl>
##  1 FSCI   6.08e 4
##  2 3WCPSF 6.17e 4
##  3 CBEY   6.29e 4
##  4 DEK    6.97e 4
##  5 RAE    6.98e 4
##  6 0419B  9.59e 4
##  7 ENTN   9.90e 4
##  8 NHLD   1.34e 5
##  9 STAQ   1.42e 5
## 10 PGUS   1.89e19
\end{verbatim}

\begin{Shaded}
\begin{Highlighting}[]
\NormalTok{SmallestAbsPerChng }\OtherTok{\textless{}{-}}\NormalTok{ finalCrisDf2 }\SpecialCharTok{\%\textgreater{}\%} \FunctionTok{arrange}\NormalTok{(pchng) }\SpecialCharTok{\%\textgreater{}\%} \FunctionTok{select}\NormalTok{(tic,pchng) }\SpecialCharTok{\%\textgreater{}\%} \FunctionTok{top\_n}\NormalTok{(}\SpecialCharTok{{-}}\DecValTok{10}\NormalTok{)}
\end{Highlighting}
\end{Shaded}

\begin{verbatim}
## Selecting by pchng
\end{verbatim}

\begin{Shaded}
\begin{Highlighting}[]
\FunctionTok{head}\NormalTok{(SmallestAbsPerChng,}\DecValTok{10}\NormalTok{)}
\end{Highlighting}
\end{Shaded}

\begin{verbatim}
## # A tibble: 10 x 2
##    tic      pchng
##    <chr>    <dbl>
##  1 MDJT.1 0      
##  2 AMRB   0.00208
##  3 UNT    0.0113 
##  4 GXP    0.0427 
##  5 SKH    0.0585 
##  6 MCO    0.0661 
##  7 CEHC   0.0896 
##  8 WASH   0.107  
##  9 KHD.Z  0.108  
## 10 GAXIQ  0.133
\end{verbatim}

\begin{enumerate}
\def\labelenumi{\alph{enumi})}
\setcounter{enumi}{1}
\tightlist
\item
  Considering the ticker values which suffered loss during crisis and
  neglecting absolute value i.e neglecting firms having an increase in
  net income . The top 10 Smallest and Largest Percentage values of
  firms are calculated respectively below .
\end{enumerate}

\begin{Shaded}
\begin{Highlighting}[]
\NormalTok{finalCrisDf3 }\OtherTok{\textless{}{-}}\NormalTok{ finalCrisDf }\SpecialCharTok{\%\textgreater{}\%} \FunctionTok{filter}\NormalTok{(pchng }\SpecialCharTok{\textless{}} \DecValTok{0}\NormalTok{)}

\NormalTok{smallstPerChng }\OtherTok{\textless{}{-}}\NormalTok{ finalCrisDf3 }\SpecialCharTok{\%\textgreater{}\%} \FunctionTok{arrange}\NormalTok{(}\FunctionTok{desc}\NormalTok{(pchng)) }\SpecialCharTok{\%\textgreater{}\%} \FunctionTok{select}\NormalTok{(tic,pchng) }\SpecialCharTok{\%\textgreater{}\%} \FunctionTok{top\_n}\NormalTok{(}\DecValTok{10}\NormalTok{)}
\end{Highlighting}
\end{Shaded}

\begin{verbatim}
## Selecting by pchng
\end{verbatim}

\begin{Shaded}
\begin{Highlighting}[]
\FunctionTok{print}\NormalTok{(smallstPerChng)}
\end{Highlighting}
\end{Shaded}

\begin{verbatim}
## # A tibble: 10 x 2
##    tic     pchng
##    <chr>   <dbl>
##  1 UNT   -0.0113
##  2 GXP   -0.0427
##  3 SKH   -0.0585
##  4 MCO   -0.0661
##  5 GAXIQ -0.133 
##  6 VLCM  -0.142 
##  7 ETCC  -0.154 
##  8 VCBI  -0.167 
##  9 TRNS  -0.229 
## 10 ASGR  -0.235
\end{verbatim}

\begin{Shaded}
\begin{Highlighting}[]
\NormalTok{largestPerChng }\OtherTok{\textless{}{-}}\NormalTok{ finalCrisDf3 }\SpecialCharTok{\%\textgreater{}\%} \FunctionTok{arrange}\NormalTok{(pchng) }\SpecialCharTok{\%\textgreater{}\%} \FunctionTok{select}\NormalTok{(tic,pchng) }\SpecialCharTok{\%\textgreater{}\%} \FunctionTok{top\_n}\NormalTok{(}\SpecialCharTok{{-}}\DecValTok{10}\NormalTok{)}
\end{Highlighting}
\end{Shaded}

\begin{verbatim}
## Selecting by pchng
\end{verbatim}

\begin{Shaded}
\begin{Highlighting}[]
\FunctionTok{print}\NormalTok{(largestPerChng)}
\end{Highlighting}
\end{Shaded}

\begin{verbatim}
## # A tibble: 10 x 2
##    tic       pchng
##    <chr>     <dbl>
##  1 PGUS   -1.89e19
##  2 NHLD   -1.34e 5
##  3 ENTN   -9.90e 4
##  4 0419B  -9.59e 4
##  5 RAE    -6.98e 4
##  6 DEK    -6.97e 4
##  7 3WCPSF -6.17e 4
##  8 BCAS   -6.04e 4
##  9 SUPR   -5.87e 4
## 10 DGNG   -5.44e 4
\end{verbatim}

\hypertarget{calculating-the-duration-it-takes-for-firms-to-recover-from-the-crisis.}{%
\subsection{Calculating the duration it takes for firms to recover from
the
crisis.}\label{calculating-the-duration-it-takes-for-firms-to-recover-from-the-crisis.}}

So we will be taking maximum value of net income during pre crisis year
and find out the duration after crisis when its highest value was
breached.

Step 1: Filtering and cleaning the data

\begin{Shaded}
\begin{Highlighting}[]
\NormalTok{dataForCrisis3 }\OtherTok{\textless{}{-}}\NormalTok{ data }\SpecialCharTok{\%\textgreater{}\%} \FunctionTok{filter}\NormalTok{(fyear }\SpecialCharTok{\textgreater{}=} \DecValTok{2004} \SpecialCharTok{\&}\NormalTok{ fyear }\SpecialCharTok{\textless{}=} \DecValTok{2014}\NormalTok{)}
\NormalTok{df5 }\OtherTok{\textless{}{-}}\NormalTok{ dataForCrisis3 }\SpecialCharTok{\%\textgreater{}\%} \FunctionTok{group\_by}\NormalTok{(tic) }\SpecialCharTok{\%\textgreater{}\%} \FunctionTok{filter}\NormalTok{(}\FunctionTok{all}\NormalTok{(}\FunctionTok{c}\NormalTok{(}\DecValTok{2004}\SpecialCharTok{:}\DecValTok{2014} \SpecialCharTok{\%in\%}\NormalTok{ fyear))) }\SpecialCharTok{\%\textgreater{}\%}\NormalTok{ ungroup}
\FunctionTok{head}\NormalTok{(df5,}\DecValTok{3}\NormalTok{)}
\end{Highlighting}
\end{Shaded}

\begin{verbatim}
## # A tibble: 3 x 3
##   fyear tic      ni
##   <int> <chr> <dbl>
## 1  2004 AIR    15.5
## 2  2005 AIR    35.2
## 3  2006 AIR    58.7
\end{verbatim}

Step 2: Calculate the maximum value of net income in pre crisis year

\begin{Shaded}
\begin{Highlighting}[]
\NormalTok{dfPre }\OtherTok{\textless{}{-}}\NormalTok{ df5 }\SpecialCharTok{\%\textgreater{}\%} \FunctionTok{filter}\NormalTok{(fyear }\SpecialCharTok{\textgreater{}=} \DecValTok{2004} \SpecialCharTok{\&}\NormalTok{ fyear }\SpecialCharTok{\textless{}=} \DecValTok{2006}\NormalTok{)}
\NormalTok{PreCrisisData2  }\OtherTok{\textless{}{-}}\NormalTok{ dfPre }\SpecialCharTok{\%\textgreater{}\%} \FunctionTok{group\_by}\NormalTok{(tic) }\SpecialCharTok{\%\textgreater{}\%} \FunctionTok{summarise}\NormalTok{( }\AttributeTok{MaxNI =} \FunctionTok{max}\NormalTok{(ni))}
\NormalTok{PreCrisisData2 }\OtherTok{\textless{}{-}}\NormalTok{ PreCrisisData2 }\SpecialCharTok{\%\textgreater{}\%} \FunctionTok{filter}\NormalTok{(MaxNI }\SpecialCharTok{!=} \StringTok{"NA"}\NormalTok{)}
\FunctionTok{head}\NormalTok{(PreCrisisData2,}\DecValTok{5}\NormalTok{)}
\end{Highlighting}
\end{Shaded}

\begin{verbatim}
## # A tibble: 5 x 2
##   tic     MaxNI
##   <chr>   <dbl>
## 1 0033A 2552   
## 2 0048A  139.  
## 3 0051A    6.33
## 4 0070A   11.8 
## 5 0071A  134.
\end{verbatim}

Step 3 : Restructing the data and summarizing it to find the duration
the firms have taken to recover from financial crisis . I have added the
column recovery in order to seperate two data frames (1 : recovered , 2:
not recovered) . Then I have seprated the two data frames . In the
recovery one I have calculated the minimum year for which the value of
recovery column is 1 for each firms and then calculated the difference
between 2008 and the min year (YearsForRecovery ) . In the another data
frame I have just selected the distinct value of firms and appended it
with a value 0 for YearsForRecovery . Then I have combined both the
dataframes and used aggregate function . In this way we won't get any
duplicate records and we will get the data of all firms who have
recovered and not recovered from the crisis .

\begin{Shaded}
\begin{Highlighting}[]
\NormalTok{df5  }\OtherTok{\textless{}{-}}\NormalTok{ df5 }\SpecialCharTok{\%\textgreater{}\%} \FunctionTok{filter}\NormalTok{(fyear}\SpecialCharTok{\textgreater{}}\DecValTok{2008}\NormalTok{)}
\NormalTok{finalnewDf }\OtherTok{\textless{}{-}} \FunctionTok{inner\_join}\NormalTok{(df5 , PreCrisisData2 , }\AttributeTok{by =} \FunctionTok{c}\NormalTok{(}\StringTok{"tic"}\NormalTok{))}
\NormalTok{finalnewDf}\SpecialCharTok{$}\NormalTok{f08 }\OtherTok{=} \DecValTok{2008}
\FunctionTok{head}\NormalTok{(finalnewDf,}\DecValTok{5}\NormalTok{)}
\end{Highlighting}
\end{Shaded}

\begin{verbatim}
## # A tibble: 5 x 5
##   fyear tic      ni MaxNI   f08
##   <int> <chr> <dbl> <dbl> <dbl>
## 1  2009 AIR    44.6  58.7  2008
## 2  2010 AIR    69.8  58.7  2008
## 3  2011 AIR    67.7  58.7  2008
## 4  2012 AIR    55    58.7  2008
## 5  2013 AIR    72.9  58.7  2008
\end{verbatim}

\begin{Shaded}
\begin{Highlighting}[]
\NormalTok{finalnewDf2 }\OtherTok{\textless{}{-}}\NormalTok{ finalnewDf }\SpecialCharTok{\%\textgreater{}\%} \FunctionTok{mutate}\NormalTok{(}\AttributeTok{recovery =} \FunctionTok{ifelse}\NormalTok{( ni }\SpecialCharTok{\textgreater{}=}\NormalTok{ MaxNI , }\DecValTok{1}\NormalTok{ , }\DecValTok{0}\NormalTok{ ))}
\FunctionTok{head}\NormalTok{(finalnewDf2,}\DecValTok{5}\NormalTok{)}
\end{Highlighting}
\end{Shaded}

\begin{verbatim}
## # A tibble: 5 x 6
##   fyear tic      ni MaxNI   f08 recovery
##   <int> <chr> <dbl> <dbl> <dbl>    <dbl>
## 1  2009 AIR    44.6  58.7  2008        0
## 2  2010 AIR    69.8  58.7  2008        1
## 3  2011 AIR    67.7  58.7  2008        1
## 4  2012 AIR    55    58.7  2008        0
## 5  2013 AIR    72.9  58.7  2008        1
\end{verbatim}

\begin{Shaded}
\begin{Highlighting}[]
\NormalTok{finalnewDf2Y }\OtherTok{\textless{}{-}}\NormalTok{ finalnewDf2 }\SpecialCharTok{\%\textgreater{}\%} \FunctionTok{filter}\NormalTok{(recovery }\SpecialCharTok{==} \DecValTok{1}\NormalTok{)}
\NormalTok{finalnewDf2N }\OtherTok{\textless{}{-}}\NormalTok{ finalnewDf2 }\SpecialCharTok{\%\textgreater{}\%} \FunctionTok{filter}\NormalTok{(recovery }\SpecialCharTok{==} \DecValTok{0}\NormalTok{)}

\NormalTok{finalnewDf2YSummary }\OtherTok{\textless{}{-}}\NormalTok{ finalnewDf2Y }\SpecialCharTok{\%\textgreater{}\%} \FunctionTok{group\_by}\NormalTok{(tic) }\SpecialCharTok{\%\textgreater{}\%} \FunctionTok{summarise}\NormalTok{(}\AttributeTok{YearsForRecovery =} \FunctionTok{min}\NormalTok{(fyear)}\SpecialCharTok{{-}}\DecValTok{2008}\NormalTok{)}
\FunctionTok{head}\NormalTok{(finalnewDf2YSummary,}\DecValTok{5}\NormalTok{)}
\end{Highlighting}
\end{Shaded}

\begin{verbatim}
## # A tibble: 5 x 2
##   tic   YearsForRecovery
##   <chr>            <dbl>
## 1 0048A                3
## 2 0070A                1
## 3 0071A                5
## 4 0100A                2
## 5 0123A                4
\end{verbatim}

\begin{Shaded}
\begin{Highlighting}[]
\NormalTok{finalnewDf2NSummary }\OtherTok{\textless{}{-}} \FunctionTok{distinct}\NormalTok{(finalnewDf2N, tic)}
\NormalTok{finalnewDf2NSummary}\SpecialCharTok{$}\NormalTok{YearsForRecovery }\OtherTok{=} \DecValTok{0}
\FunctionTok{head}\NormalTok{(finalnewDf2NSummary,}\DecValTok{5}\NormalTok{)}
\end{Highlighting}
\end{Shaded}

\begin{verbatim}
## # A tibble: 5 x 2
##   tic   YearsForRecovery
##   <chr>            <dbl>
## 1 AIR                  0
## 2 AAL                  0
## 3 CECE                 0
## 4 AVX                  0
## 5 PNW                  0
\end{verbatim}

\begin{Shaded}
\begin{Highlighting}[]
\NormalTok{finalData }\OtherTok{\textless{}{-}} \FunctionTok{rbind}\NormalTok{(finalnewDf2YSummary ,finalnewDf2NSummary )}
\NormalTok{dataForRecoveryCrisis }\OtherTok{\textless{}{-}} \FunctionTok{aggregate}\NormalTok{(finalData}\SpecialCharTok{$}\NormalTok{YearsForRecovery, }\AttributeTok{by =} \FunctionTok{list}\NormalTok{(finalData}\SpecialCharTok{$}\NormalTok{tic), }\AttributeTok{FUN =}\NormalTok{ sum) }
\NormalTok{dataForRecoveryCrisis }\OtherTok{\textless{}{-}}\NormalTok{ dataForRecoveryCrisis }\SpecialCharTok{\%\textgreater{}\%} \FunctionTok{mutate}\NormalTok{(}\AttributeTok{YearsRecover =} \FunctionTok{ifelse}\NormalTok{(x }\SpecialCharTok{\textgreater{}} \DecValTok{0}\NormalTok{ , x ,}\StringTok{\textquotesingle{}NA\textquotesingle{}}\NormalTok{))}
\NormalTok{dataForRecoveryCrisis }\OtherTok{=} \FunctionTok{rename}\NormalTok{(dataForRecoveryCrisis , }\StringTok{"tic"} \OtherTok{=}\NormalTok{ Group}\FloatTok{.1}\NormalTok{ )}
\NormalTok{dataForRecoveryCrisis   }\OtherTok{\textless{}{-}} \FunctionTok{subset}\NormalTok{(dataForRecoveryCrisis  , }\AttributeTok{select =} \SpecialCharTok{{-}}\NormalTok{x)}
\FunctionTok{head}\NormalTok{(dataForRecoveryCrisis ,}\DecValTok{10}\NormalTok{)}
\end{Highlighting}
\end{Shaded}

\begin{verbatim}
##      tic YearsRecover
## 1  0033A           NA
## 2  0048A            3
## 3  0051A           NA
## 4  0070A            1
## 5  0071A            5
## 6  0079A           NA
## 7  0084A           NA
## 8  0100A            2
## 9  0123A            4
## 10 0124A           NA
\end{verbatim}

\begin{Shaded}
\begin{Highlighting}[]
\NormalTok{dataForRecoveryCrisisSum }\OtherTok{\textless{}{-}}\NormalTok{ dataForRecoveryCrisis }\SpecialCharTok{\%\textgreater{}\%} \FunctionTok{filter}\NormalTok{(YearsRecover }\SpecialCharTok{!=} \StringTok{\textquotesingle{}NA\textquotesingle{}}\NormalTok{) }\SpecialCharTok{\%\textgreater{}\%} \FunctionTok{mutate}\NormalTok{(}\AttributeTok{yearNum =}  \FunctionTok{as.numeric}\NormalTok{(YearsRecover))}
\NormalTok{dataForRecoveryCrisisSum  }\OtherTok{\textless{}{-}} \FunctionTok{subset}\NormalTok{(dataForRecoveryCrisisSum , }\AttributeTok{c=} \SpecialCharTok{{-}}\NormalTok{ YearsRecover)}
\NormalTok{dataForRecoveryCrisisSum  }\OtherTok{\textless{}{-}}\NormalTok{ dataForRecoveryCrisisSum   }\SpecialCharTok{\%\textgreater{}\%} \FunctionTok{summarise}\NormalTok{(}\AttributeTok{mean\_dd =} \FunctionTok{mean}\NormalTok{(yearNum), }\AttributeTok{sd\_dd =} \FunctionTok{sd}\NormalTok{(yearNum), }\AttributeTok{n =} \FunctionTok{n}\NormalTok{())}
\FunctionTok{print}\NormalTok{(dataForRecoveryCrisisSum)}
\end{Highlighting}
\end{Shaded}

\begin{verbatim}
##    mean_dd    sd_dd    n
## 1 2.146608 1.444468 2742
\end{verbatim}

There are many companies who took long time to recover . Based on
standard mean it can be concluded that after crisis it almost took 2.25
years for company to breach its highest level during which was
calculated during pre crisis period .

\end{document}
